% !TeX program = XeLaTeX
% !TeX root = ../nAmA.tex
\chapt{शक्त्यष्टोत्तरशतदिव्यस्थानीयनामावलिः}
\begin{multicols}{2}
\begin{flushleft}
वाराणस्यां विशालाक्ष्यै~नमः\\
नैमिषे लिङ्गधारिण्यै~नमः\\
प्रयागे ललितादेव्यै~नमः\\
गन्धमादने कामाक्ष्यै~नमः\\
मानसे कुमुदायै~नमः\\
अम्बरे विश्वकायायै~नमः\\
गोमन्ते गोमत्यै~नमः\\
मन्दरे कामचारिण्यै~नमः\\
चैत्ररथे मदोत्कटायै~नमः\\
हस्तिनापुरे जयन्त्यै~नमः\hfill\devanumber{10}\\
कान्यकुब्जे गौर्यै~नमः\\
मलयपर्वते रम्भायै~नमः\\
एकाम्रके कीर्तिमत्यै~नमः\\
विश्वे विश्वेश्वर्यै~नमः\\
पुष्करे पुरुहूतायै~नमः\\
केदारे मार्गदायिन्यै~नमः\\
हिमवतःपृष्ठे नन्दायै~नमः\\
गोकर्णे भद्रकर्णिकायै~नमः\\
स्थानेश्वरे भवान्यै~नमः\\
बिल्वके बिल्वपत्रिकायै~नमः\hfill\devanumber{20}\\
श्रीशैले माधव्यै~नमः\\
भद्रेश्वरे भद्रायै~नमः\\
वराहशैले जयायै~नमः\\
कमलालये कमलायै~नमः\\
रुद्रकोट्यां रुद्राण्यै~नमः\\
कालञ्जरे गिरौ काल्यै~नमः\\
महालिङ्गे कपिलायै~नमः\\
मर्कोटे मुकुटेश्वर्यै~नमः\\
शालग्रामे महादेव्यै~नमः\\
शिवलिङ्गे जलप्रियायै~नमः\hfill\devanumber{30}\\
मायापुर्यां कुमार्यै~नमः\\
सन्ताने ललितायै~नमः\\
सहस्राक्षे उत्पलाक्ष्यै~नमः\\
कमलाक्षे महोत्पलायै~नमः\\
गङ्गायां मङ्गलायै~नमः\\
पुरुषोत्तमे विमलायै~नमः\\
विपाशायां अमोघाक्ष्यै~नमः\\
पुण्ड्रवर्धने पाटलायै~नमः\\
सुपार्श्वे नारायण्यै~नमः\\
विकूटे भद्रसुन्दर्यै~नमः\hfill\devanumber{40}\\
विपुले विपुलायै~नमः\\
मलयाचले कल्याण्यै~नमः\\
कोटितीर्थे कोटव्यै~नमः\\
माधवे वने सुगन्धायै~नमः\\
कुब्जाम्रके त्रिसन्ध्यायै~नमः\\
गङ्गाद्वारे रतिप्रियायै~नमः\\
शिवकुण्डे सुनन्दायै~नमः\\
देविकातटे नन्दिन्यै~नमः\\
द्वारवत्यां रुक्मिण्यै~नमः\\
वृन्दावने वने राधायै~नमः\hfill\devanumber{50}\\
मथूरायां देविकायै~नमः\\
पाताले परमेश्वर्यै~नमः\\
चित्रकूटे सीतायै~नमः\\
विन्ध्ये विन्ध्याधिवासिन्यै~नमः\\
सह्याद्रौ एकवीरायै~नमः\\
हरिश्चन्द्रे चन्द्रिकायै~नमः\\
रामतीर्थे रमणायै~नमः\\
यमुनायां मृगावत्यै~नमः\\
करवीरे महालक्ष्म्यै~नमः\\
विनायके उमादेव्यै~नमः\hfill\devanumber{60}\\
वैद्यनाथे अरोगायै~नमः\\
महाकाले महेश्वर्यै~नमः\\
उष्णतीर्थेषु अभयायै~नमः\\
विन्ध्यकन्दरे अमृतायै~नमः\\
माण्डव्ये माण्डव्यै~नमः\\
महेश्वरे पुरे स्वाहायै~नमः\\
छागलाण्डे प्रचण्डायै~नमः\\
मकरन्दके चण्डिकायै~नमः\\
सोमेश्वरे वरारोहायै~नमः\\
प्रभासे पुष्करावत्यै~नमः\hfill\devanumber{70}\\
सरस्वत्यां पारावारतटे देवमात्रे~नमः\\
महालये महाभागायै~नमः\\
पयोष्ण्यां पिङ्गलेश्वर्यै~नमः\\
कृतशौचे सिंहिकायै~नमः\\
कार्त्तिकेये यशस्कर्यै~नमः\\
उत्पलावर्तके लोलायै~नमः\\
शोणसङ्गमे सुभद्रायै~नमः\\
सिद्धपुरे मात्रे लक्ष्म्यै~नमः\\
भरताश्रमे अङ्गनायै~नमः\\
जालन्धरे विश्वमुख्यै~नमः\hfill\devanumber{80}\\
किष्किन्धपर्वते तारायै~नमः\\
देवदारुवने पुष्ट्यै~नमः\\
काश्मीर मण्डले मेधायै~नमः\\
हिमाद्रौ भीमादेव्यै~नमः\\
विश्वेश्वरे पुष्ट्यै~नमः\\
कपालमोचने शुद्ध्यै~नमः\\
कायावरोहणे मात्रे~नमः\\
शङ्खोद्धारे ध्वन्यै~नमः\\
पिण्डारके धृत्यै~नमः\\
चद्रभागायां कालायै~नमः\hfill\devanumber{90}\\
अच्चोदे शिवकारिण्यै~नमः\\
वेणायां अमृतायै~नमः\\
बदर्यां उर्वश्यै~नमः\\
उत्तरकुरौ औषध्यै~नमः\\
कृशद्वीपे कुशोदकायै~नमः\\
हेमकूटे मन्मथायै~नमः\\
मुकुटे सत्यवादिन्यै~नमः\\
अश्वत्थे वन्दनीयायै~नमः\\
वैश्रवणालये निधये~नमः\\
वेदवदने गायत्र्यै~नमः\hfill\devanumber{100}\\
शिवसन्निधौ पार्वत्यै~नमः\\
देवलोके इन्द्राण्यै~नमः\\
ब्रह्मास्येषु सरस्वत्यै~नमः\\
सूर्यबिम्बे प्रभायै~नमः\\
मातॄणां वैष्णव्यै~नमः\\
सतीनां अरुन्धत्यै~नमः\\
रामासु तिलोत्तमायै~नमः\\
सर्वशरीरिणां चित्ते ब्रह्मकलानामशक्त्यै~नमः\\
\end{flushleft}
\end{multicols}
॥इति श्री मत्स्यमहापुराणे श्री शक्त्यष्टोत्तर\-शतदिव्यस्थानीयनामावलिः सम्पूर्णा॥
