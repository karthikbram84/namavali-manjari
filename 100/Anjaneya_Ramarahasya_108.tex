% !TeX program = XeLaTeX
% !TeX root = ../nAmA.tex
\chapt{हनुमदष्टोत्तरशतनामावलिः}
\begin{multicols}{2}
\begin{flushleft}
हनुमते~नमः\\
अञ्जना-सूनवे~नमः\\
धीमते~नमः\\
केसरि-नन्दनाय~नमः\\
वातात्मजाय~नमः\\
वर-गुणाय~नमः\\
वानरेन्द्राय~नमः \\
विरोचनाय~नमः\\
सुग्रीव-सचिवाय~नमः\\
श्रीमते~नमः \hfill १०\\
सूर्य-शिष्याय~नमः\\
सुख-प्रदाय~नमः \\
ब्रह्म-दत्त-वराय~नमः\\
ब्रह्म-भूताय~नमः\\
ब्रह्मर्षि-सन्नुताय~नमः\\
जितेन्द्रियाय~नमः\\
जितारातये~नमः\\
राम-दूताय~नमः\\
रणोत्कटाय~नमः \\
सञ्जीवनी-समाहर्त्रे~नमः \hfill २०\\
सर्व-सैन्य-प्रहर्षकाय~नमः\\
रावणाकम्प्य-सौमित्रि-नयन-स्फुट-भक्तिमते~नमः\\
अशोक-वनिका-च्छेदिने~नमः\\
सीता-वात्सल्य-भाजनाय~नमः\\
विषीदद्-भूमि-तनयार्पित-रामाङ्गुलीयकाय~नमः\\
चूडामणि-समानेत्रे~नमः\\
राम-दुःखापहारकाय~नमः\\
अक्ष-हन्त्रे~नमः\\
विक्षतारये~नमः\\
तृणीकृत-दशाननाय~नमः \hfill ३०\\
कुल्या-कल्प-महाम्भोधये~नमः\\
सिंहिका-प्राण-नाशनाय~नमः\\
सुरसा-विजयोपाय-वेत्त्रे~नमः\\
सुर-वरार्चिताय~नमः\\
जाम्बवन्नुत-माहात्म्याय~नमः\\
जीविताहत-लक्ष्मणाय~नमः\\
जम्बुमालि-रिपवे~नमः\\
जम्भ-वैरि-साध्वस-नाशनाय~नमः\\
अस्त्रावध्याय~नमः\\
राक्षसारये~नमः \hfill ४०\\
सेनापति-विनाशनाय~नमः\\
लङ्कापुर-प्रदग्ध्रे~नमः\\
वालानल-सुशीतलाय~नमः\\
वानर-प्राण-सन्दात्रे~नमः\\
वालि-सूनु-प्रियङ्कराय~नमः\\
महारूप-धराय~नमः\\
मान्याय~नमः\\
भीमाय~नमः\\
भीम-पराक्रमाय~नमः\\
भीम-दर्प-हराय~नमः \hfill ५०\\
भक्त-वत्सलाय~नमः\\
भर्त्सिताशराय~नमः\\
रघु-वंश-प्रिय-कराय~नमः\\
रण-धीराय~नमः\\
रयाकराय~नमः\\
भरतार्पित-सन्देशाय~नमः\\
भगवच्छ्लिष्ट-विग्रहाय~नमः\\
अर्जुन-ध्वज-वासिने~नमः\\
तर्जिताशर-नायकाय~नमः\\
महते~नमः \hfill ६०\\
महा-मधुर-वाचे~नमः\\
महात्मने~नमः\\
मातरिश्व-जाय~नमः\\
मरुन्नुताय~नमः\\
महोदार-गुणाय~नमः\\
मधु-वन-प्रियाय~नमः\\
महा-धैर्याय~नमः\\
महा-वीर्याय~नमः\\
मिहिराधिक-कान्तिमते~नमः\\
अन्नदाय~नमः \hfill ७०\\
वसुदाय~नमः \\
वाग्मिने~नमः\\
ज्ञान-दाय~नमः\\
वत्सलाय~नमः\\
वशिने~नमः\\
वशीकृताखिल-जगते~नमः\\
वरदाय~नमः\\
वानराकृतये~नमः\\
भिक्षु-रूप-प्रतिच्छन्नाय~नमः\\
अभीति-दाय~नमः \hfill ८०\\
भीति-वर्जिताय~नमः \\
भूमी-धर-हराय~नमः\\
भूति-दायकाय~नमः\\
भूत-सन्नुताय~नमः\\
भुक्ति-मुक्ति-प्रदाय~नमः\\
भूम्ने~नमः\\
भुज-निर्जित-राक्षसाय~नमः\\
वाल्मीकि-स्तुत-माहात्म्याय~नमः\\
विभीषण-सुहृदे~नमः \\
विभवे~नमः  \hfill ९०\\
अनुकम्पा-निधये~नमः\\
पम्पा-तीर-चारिणे~नमः\\
प्रतापवते~नमः\\
ब्रह्मास्त्र-हत-रामादि-जीवनाय~नमः \\
ब्रह्म-वत्सलाय~नमः \\
जय-वार्ताहराय~नमः\\
जेत्रे~नमः \\
जानकी-शोक-नाशनाय~नमः\\
जानकी-राम-साहित्य-कारिणे~नमः\\
जन-सुख-प्रदाय~नमः \hfill १००\\
बहु-योजन-गन्त्रे~नमः\\
बल-वीर्य-गुणाधिकाय~नमः \\
रावणालय-मर्दिने~नमः\\
राम-पादाब्ज-वाहकाय~नमः  \\
राम-नाम-लसद्-वक्त्राय~नमः \\
रामायण-कथादृताय~नमः\\
राम-स्वरूप-विलसन्मानसाय~नमः\\
राम-वल्लभाय~नमः \hfill १०८
\end{flushleft}
\end{multicols}
\threelineshloka*
{इत्थम् अष्टोत्तर-शतं नाम्नां वातात्मजस्य यः}
{अनुसन्ध्यं पठेत् तस्य मारुतिः सम्प्रसीदति}
{प्रसन्ने मारुतौ रामो भुक्ति-मुक्ती प्रयच्छति} 


॥इति ब्रह्मयामले रामरहस्ये प्रोक्ता श्री-हनुमद्-अष्टोत्तरशत-नामावलिः सम्पूर्णा॥

