% !TeX program = XeLaTeX
% !TeX root = ../nAmA.tex
\chapt{सीताष्टोत्तरशतनामावलिः}
\begin{multicols}{2}
\begin{flushleft}
सीतायै~नमः\\
सीर-ध्वज-सुतायै~नमः\\
सीमातीत-गुणोज्ज्वलायै~नमः\\
सौन्दर्य-सार-सर्वस्व-भूतायै~नमः\\
सौभाग्य-दायिन्यै~नमः\\
देव्यै~नमः\\
देवार्चित-पदायै~नमः \\
दिव्यायै~नमः\\
दशरथ-स्नुषायै~नमः\\
रामायै~नमः \hfill १०\\
राम-प्रियायै~नमः\\
रम्यायै~नमः  \\
राकेन्दु-वदनोज्ज्वलायै~नमः\\
वीर्य-शुल्कायै~नमः\\
वीर-पत्न्यै~नमः\\
वियन्मध्यायै~नमः\\
वर-प्रदायै~नमः\\
पति-व्रतायै~नमः\\
पङ्क्ति-कण्ठ-नाशिन्यै~नमः \\
पावन-स्मृतये~नमः \hfill २०\\
वन्दारु-वत्सलायै~नमः\\
वीर-मात्रे~नमः\\
वृत-रघूत्तमायै~नमः\\
सम्पत्-कर्यै~नमः\\
सदा-तुष्टायै~नमः\\
साक्षिण्यै~नमः\\
साधु-सम्मतायै~नमः\\
नित्यायै~नमः\\
नियत-संस्थानायै~नमः\\
नित्यानन्दायै~नमः \hfill ३०\\
नुति-प्रियायै~नमः\\
पृथ्व्यै~नमः\\
पृथ्वी-सुतायै~नमः\\
पुत्र-दायिन्यै~नमः\\
प्रकृत्यै परस्यै~नमः\\
हनूमत्-स्वामिन्यै~नमः\\
हृद्यायै~नमः\\
हृदय-स्थायै~नमः\\
हताशुभायै~नमः\\
हंस-युक्तायै~नमः \hfill ४०\\
हंस-गतये~नमः\\
हर्ष-युक्तायै~नमः\\
हताशरायै~नमः\\
सार-रूपायै~नमः\\
सार-वचसे~नमः\\
साध्व्यै~नमः\\
सरमा-प्रियायै~नमः\\
त्रिलोक-वन्द्यायै~नमः\\
त्रिजटा-सेव्यायै~नमः\\
त्रिपथ-गार्चिन्यै~नमः \hfill ५०\\
त्राण-प्रदायै~नमः\\
त्रात-काकायै~नमः\\
तृणी-कृत-दशाननायै~नमः\\
अनसूया-अङ्गरागाङ्कायै~नमः\\
अनसूयायै~नमः\\
सूरि-वन्दितायै~नमः\\
अशोक-वनिका-स्थानायै~नमः\\
अशोकायै~नमः\\
शोक-विनाशिन्यै~नमः\\
सूर्य-वंश-स्नुषायै~नमः \hfill ६०\\
सूर्य-मण्डलान्तस्स्थ-वल्लभायै~नमः\\
श्रुत-मात्राघ-हरणायै~नमः\\
श्रुति-सन्निहितेक्षणायै~नमः\\
पुष्प-प्रियायै~नमः\\
पुष्पक-स्थायै~नमः\\
पुण्य-लभ्यायै~नमः\\
पुरातन्यै~नमः\\
पुरुषार्थ-प्रदायै~नमः\\
पूज्यायै~नमः\\
पूत-नाम्ने~नमः \hfill ७०\\
परन्तपायै~नमः \\
पद्म-प्रियायै~नमः\\
पद्म-हस्तायै~नमः\\
पद्मायै~नमः\\
पद्म-मुख्यै~नमः\\
शुभायै~नमः\\
जन-शोक-हरायै~नमः\\
जन्म-मृत्यु-शोक-विनाशिन्यै~नमः\\
जगद्-रूपायै~नमः\\
जगद्-वन्द्यायै~नमः \hfill ८०\\
जय-दायै~नमः \\
जनकात्मजायै~नमः\\
नाथनीय-कटाक्षायै~नमः\\
नाथायै~नमः\\
नाथैक-तत्परायै~नमः\\
नक्षत्र-नाथ-वदनायै~नमः\\
नष्ट-दोषायै~नमः\\
नयावहायै~नमः\\
वह्नि-पाप-हरायै~नमः \\
वह्नि-शैत्य-कृते~नमः  \hfill ९०\\
वृद्धि-दायिन्यै~नमः\\
वाल्मीकि-गीत-विभवायै~नमः\\
वचोतीतायै~नमः\\
वराङ्गनायै~नमः \\
भक्ति-गम्यायै~नमः \\
भव्य-गुणायै~नमः\\
भात्र्यै~नमः \\
भरत-वन्दितायै~नमः\\
सुवर्णाङ्ग्यै~नमः\\
सुखकर्यै~नमः \hfill १००\\
सुग्रीवाङ्गद-सेवितायै~नमः\\
वैदेह्यै~नमः \\
विनताघौघ-नाशिन्यै~नमः\\
विधि-वन्दितायै~नमः  \\
लोक-मात्रे~नमः \\
लोचनान्त-स्थित-कारुण्य-सागरायै~नमः\\
कृताकृत-जगद्धेतवे~नमः\\
कृत-राज्याभिषेककायै~नमः \hfill १०८
\end{flushleft}
\end{multicols}

\twolineshloka*
{इदम् अष्टोत्तर-शतं सीता-नाम्नां तु या वधूः}
{भक्ति-युक्ता पठेत् सा तु पुत्र-पौत्रादि-नन्दिता}

\twolineshloka*
{धन-धान्य-समृद्धा च दीर्घ-सौभाग्य-दर्शिनी}
{पुंसाम् अपि स्तोत्रम् इदं पठनात् सर्व-सिद्धि-दम्}


॥इति ब्रह्मयामले रामरहस्ये प्रोक्ता श्री-सीता-अष्टोत्तरशत-नामावलिः सम्पूर्णा॥
