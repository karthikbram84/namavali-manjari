% !TeX program = XeLaTeX
% !TeX root = ../nAmA.tex
\chapt{रामाष्टोत्तरशतनामावलिः}
\begin{multicols}{2}
\begin{flushleft}
रामाय~नमः\\
रावण-संहार-कृत-मानुष-विग्रहाय~नमः\\
कौसल्या-सुकृत-व्रात-फलाय~नमः\\
दशरथात्मजाय~नमः\\
लक्ष्मणार्चित-पादाब्जाय~नमः\\
सर्व-लोक-प्रियङ्कराय~नमः\\
साकेत-वासि-नेत्राब्ज-सम्प्रीणन-दिवाकराय~नमः \\
विश्वामित्र-प्रियाय~नमः\\
शान्ताय~नमः\\
ताटका-ध्वान्त-भास्कराय~नमः \hfill १०\\
सुबाहु-राक्षस-रिपवे~नमः\\
कौशिकाध्वर-पालकाय~नमः \\
अहल्या-पाप-संहर्त्रे~नमः\\
जनकेन्द्र-प्रियातिथये~नमः\\
पुरारि-चाप-दलनाय~नमः\\
वीर-लक्ष्मी-समाश्रयाय~नमः\\
सीता-वरण-माल्याढ्याय~नमः\\
जामदग्न्य-मदापहाय~नमः\\
वैदेही-कृत-शृङ्गाराय~नमः \\
पितृ-प्रीति-विवर्धनाय~नमः \hfill २०\\
ताताज्ञोत्सृष्ट-हस्त-स्थ-राज्याय~नमः\\
सत्य-प्रतिश्रवाय~नमः\\
तमसा-तीर-संवासिने~नमः\\
गुहानुग्रह-तत्पराय~नमः\\
सुमन्त्र-सेवित-पदाय~नमः\\
भरद्वाज-प्रियातिथये~नमः\\
चित्रकूट-प्रिय-स्थानाय~नमः\\
पादुका-न्यस्त-भू-भराय~नमः\\
अनसूया-अङ्गरागाङ्क-सीता-साहित्य-शोभिताय~नमः\\
दण्डकारण्य-सञ्चारिणे~नमः \hfill ३०\\
विराध-स्वर्ग-दायकाय~नमः\\
रक्षः-कालान्तकाय~नमः\\
सर्व-मुनि-सङ्घ-मुदावहाय~नमः\\
प्रतिज्ञात-आशर-वधाय~नमः\\
शरभङ्ग-गति-प्रदाय~नमः\\
अगस्त्यार्पित-बाणास-खड्ग-तूणीर-मण्डिताय~नमः\\
प्राप्त-पञ्चवटी-वासाय~नमः\\
गृध्रराज-सहायवते~नमः\\
कामि-शूर्पणखा-कर्ण-नास-च्छेद-नियामकाय~नमः\\
खरादि-राक्षस-त्रातखण्डनावितसञ्जनाय~नमः \hfill ४०\\
सीता-संश्लिष्ट-कायाभा-जित-विद्युद्-युताम्बुदाय~नमः\\
मारीच-हन्त्रे~नमः\\
मायाढ्याय~नमः\\
जटायुर्-मोक्ष-दायकाय~नमः\\
कबन्ध-बाहु-लवनाय~नमः\\
शबरी-प्रार्थितातिथये~नमः\\
हनुमद्-वन्दित-पदाय~नमः\\
सुग्रीव-सुहृदेऽव्ययाय~नमः\\
दैत्य-कङ्काल-विक्षेपिणे~नमः\\
सप्त-साल-प्रभेदकाय~नमः \hfill ५०\\
एकेषु-हत-वालिने~नमः\\
तारा-संस्तुत-सद्गुणाय~नमः\\
कपीन्द्री-कृत-सुग्रीवाय~नमः\\
सर्व-वानर-पूजिताय~नमः\\
वायु-सूनु-समानीत-सीता-सन्देश-नन्दिताय~नमः\\
जैत्र-यात्रोद्यताय~नमः\\
जिष्णवे~नमः\\
विष्णु-रूपाय~नमः\\
निराकृतये~नमः\\
कम्पिताम्भोनिधये~नमः \hfill ६०\\
सम्पत्-प्रदाय~नमः\\
सेतु-निबन्धनाय~नमः\\
लङ्का-विभेदन-पटवे~नमः\\
निशाचर-विनाशकाय~नमः\\
कुम्भकर्णाख्य-कुम्भीन्द्र-मृगराज-पराक्रमाय~नमः\\
मेघनाद-वधोद्युक्त-लक्ष्मणास्त्र-बलप्रदाय~नमः\\
दशग्रीवान्धतामिस्र-प्रमापण-प्रभाकराय~नमः\\
इन्द्रादि-देवता-स्तुत्याय~नमः\\
चन्द्राभ-मुख-मण्डलाय~नमः\\
विभीषणार्पित-निशाचर-राज्याय~नमः \hfill ७०\\
वृष-प्रियाय~नमः \\
वैश्वानर-स्तुत-गुणावनिपुत्री-समागताय~नमः\\
पुष्पक-स्थान-सुभगाय~नमः\\
पुण्यवत्-प्राप्य-दर्शनाय~नमः\\
राज्याभिषिक्ताय~नमः\\
राजेन्द्राय~नमः\\
राजीव-सदृशेक्षणाय~नमः\\
लोक-तापापहर्त्रे~नमः\\
धर्म-संस्थापनोद्यताय~नमः\\
शरण्याय~नमः \hfill ८०\\
कीर्तिमते~नमः \\
नित्याय वदान्याय~नमः\\
करुणार्णवाय~नमः\\
संसार-सिन्धु-सम्मग्न-तारकाख्या-महोज्ज्वलाय~नमः\\
मधुराय~नमः\\
मधुरोक्तये~नमः\\
मधुरा-नायकाग्रजाय~नमः\\
शम्बूक-दत्त-स्वर्लोकाय~नमः\\
शम्बराराति-सुन्दराय~नमः \\
अश्वमेध-महायाजिने~नमः  \hfill ९०\\
वाल्मीकि-प्रीतिमते~नमः\\
वशिने~नमः\\
स्वयं रामायण-श्रोत्रे~नमः\\
पुत्र-प्राप्ति-प्रमोदिताय~नमः \\
ब्रह्मादि-स्तुत-माहात्म्याय~नमः \\
ब्रह्मर्षि-गण-पूजिताय~नमः\\
वर्णाश्रम-रताय~नमः \\
वर्णाश्रम-धर्म-नियामकाय~नमः\\
रक्षा-पराय~नमः\\
राज-वंश-प्रतिष्ठापन-तत्पराय~नमः \hfill १००\\
गन्धर्व-हिंसा-संहारिणे~नमः\\
धृतिमते~नमः \\
दीन-वत्सलाय~नमः\\
ज्ञानोपदेष्ट्रे~नमः  \\
वेदान्त-वेद्याय~नमः \\
भक्त-प्रियङ्कराय~नमः\\
वैकुण्ठ-वासिने~नमः\\
चराचर-विमुक्ति-दाय~नमः \hfill १०८
\end{flushleft}
\end{multicols}
॥इति ब्रह्मयामले रामरहस्ये प्रोक्ता श्री-राम-अष्टोत्तरशत-नामावलिः सम्पूर्णा॥
