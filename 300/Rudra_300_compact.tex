% !TeX program = XeLaTeX
% !TeX root = ../nAmA.tex
\sect{श्रीरुद्रनाम त्रिशती}
नमो॒ हिर॑ण्यबाहवे॒ नमः॑। से॒ना॒न्ये॑  नमः॑। \\
दि॒शां च॒ पत॑ये॒ नमः॑। नमो॑ वृ॒क्षेभ्यो॒ नमः॑। \\
हरि॑केशेभ्यो॒  नमः॑। प॒शू॒नां पत॑ये॒  नमः॑। \\
नमः॑ स॒स्पिञ्ज॑राय॒ नमः॑। त्विषी॑मते॒ नमः॑। \\
प॒थी॒नां पत॑ये॒ नमः॑। नमो॑ बभ्लु॒शाय॒ नमः॑। \\
वि॒व्या॒धिने॒ नमः॑। अन्ना॑नां॒ पत॑ये॒ नमः॑।  \\
नमो॒ हरि॑केशाय॒ नमः॑। उ॒प॒वी॒तिने॒ नमः॑। \\
पु॒ष्टानां॒ पत॑ये॒ नमः॑। नमो॑ भ॒वस्य॑ हे॒त्यै नमः॑। \\
जग॑तां॒ पत॑ये॒ नमः॑। नमो॑ रु॒द्राय॒ नमः॑। \\
आ॒त॒ता॒विने॒ नमः॑। क्षेत्रा॑णां॒ पत॑ये॒ नमः॑।  \\
नमः॑ सू॒ताय॒ नमः॑। अह॑न्त्याय॒ नमः॑। \\
वना॑नां॒  पत॑ये॒ नमः॑। नमो॒ रोहि॑ताय॒ नमः॑। \\
स्थ॒पत॑ये नमः॑। वृ॒क्षाणां॒ पत॑ये॒ नमः॑। \\
नमो॑ म॒न्त्रिणे॒ नमः॑। वा॒णि॒जाय॒ नमः॑। \\
कक्षा॑णां॒ पत॑ये॒ नमः॑। नमो॑ भुव॒न्तये॒ नमः॑। \\
वा॒रि॒व॒स्कृ॒ताय॒ नमः॑। ओष॑धीनां॒ पत॑ये॒ नमः॑। \\
नम॑ उ॒च्चैर्घो॑षाय॒ नमः॑। आ॒क्र॒न्दय॑ते॒ नमः॑। \\
प॒त्ती॒नां पत॑ये॒ नमः॑। नमः॑ कृथ्स्नवी॒ताय॒ नमः॑। \\
धाव॑ते॒ नमः॑। सत्त्व॑नां॒ पत॑ये॒ नमः॑॥\\

नमः॒ सह॑मानाय॒ नमः॑। नि॒व्या॒धिने॒ नमः॑। \\
आ॒व्या॒धिनी॑नां॒ पत॑ये॒ नमः॑। नमः॑ ककु॒भाय॒ नमः॑। \\
नि॒ष॒ङ्गिणे॒ नमः॑। स्ते॒नानां॒ पत॑ये॒ नमः॑। \\
नमो॑ निष॒ङ्गिणे॒ नमः॑। इ॒षु॒धि॒मते॒ नमः॑। \\
तस्क॑राणां॒ पत॑ये॒ नमः॑। नमो॒ वञ्च॑ते॒ नमः॑। \\
प॒रि॒वञ्च॑ते॒ नमः॑। स्ता॒यू॒नां पत॑ये॒ नमः॑। \\
नमो॑ निचे॒रवे॒ नमः॑। प॒रि॒च॒राय॒ नमः॑। \\
अर॑ण्यानां॒ पत॑ये॒ नमः॑। नमः॑ सृका॒विभ्यो॒ नमः॑। \\
जिघाꣳ॑सद्भ्यो॒ नमः॑। मु॒ष्ण॒तां पत॑ये॒ नमः॑। \\
नमो॑ऽसि॒मद्भ्यो॒ नमः॑। नक्तं॒ चर॑द्भ्यो॒ नमः॑। \\
प्र॒कृ॒न्तानां॒ पत॑ये॒ नमः॑। नम॑ उष्णी॒षिने॒ नमः॑। \\
गि॒रि॒च॒राय॒ नमः॑। कु॒लु॒ञ्चानां॒ पत॑ये॒  नमः॑। \\
नम॒ इषु॑मद्भ्यो॒  नमः॑। ध॒न्वा॒विभ्य॑श्च॒  नमः॑। वो॒  नमः॑। \\
नम॑ आतन्वा॒नेभ्यो॒ नमः॑। प्र॒ति॒दधा॑नेभ्यश्च॒  नमः॑। वो॒ नमः॑।\\
नम॑ आ॒यच्छ॑द्भ्यो॒  नमः॑।  वि॒सृ॒जद्भ्य॑श्च॒  नमः॑।  वो॒ नमः॑।\\
नमोऽस्य॑द्भ्यो॒  नमः॑। विध्य॑द्भ्यश्च॒  नमः॑।  वो॒ नमः॑।\\
नम॒ आसी॑नेभ्यो॒  नमः॑। शया॑नेभ्यश्च॒  नमः॑। वो॒ नमः॑।\\
नमः॑ स्व॒पद्भ्यो॒  नमः॑। जाग्र॑द्भ्यश्च॒  नमः॑।  वो॒ नमः॑।\\
नम॒स्तिष्ठ॑द्भ्यो॒  नमः॑। धाव॑द्भ्यश्च॒  नमः॑।  वो॒ नमः॑।\\
नमः॑ स॒भाभ्यो॒  नमः॑। स॒भाप॑तिभ्यश्च॒ नमः॑। वो॒ नमः॑।\\
नमो॒ अश्वे᳚भ्यो॒  नमः॑। अश्व॑पतिभ्यश्च॒  नमः॑। वो॒ नमः॑॥\\


नम॑ आव्या॒धिनी᳚भ्यो॒  नमः॑। वि॒विध्य॑न्तीभ्यश्च॒  नमः॑।  वो॒ नमः॑। \\
नम॒ उग॑णाभ्यो॒ नमः॑। तृ॒ꣳ॒ह॒तीभ्य॑श्च॒ नमः॑। वो॒ नमः॑। \\
नमो॑ गृ॒थ्सेभ्यो॒ नमः॑। गृ॒थ्सप॑तिभ्यश्च॒ नमः॑। वो॒ नमः॑। \\
नमो॒ व्राते᳚भ्यो॒ नमः॑। व्रात॑पतिभ्यश्च॒ नमः॑। वो॒ नमः॑। \\
नमो॑ ग॒णेभ्यो॒ नमः॑। ग॒णप॑तिभ्यश्च॒ नमः॑। वो॒ नमः॑।\\
नमो॒ विरू॑पेभ्यो॒ नमः॑। वि॒श्वरू॑पेभ्यश्च॒ नमः॑। वो॒ नमः॑। \\
नमो॑ म॒हद्भ्यो॒ नमः॑। क्षु॒ल्ल॒केभ्य॑श्च॒ नमः॑। वो॒ नमः॑।\\
नमो॑ र॒थिभ्यो॒ नमः॑। अ॒र॒थेभ्य॑श्च॒ नमः॑। वो॒ नमः॑। \\
नमो॒ रथे᳚भ्यो॒ नमः॑। रथ॑पतिभ्यश्च॒ नमः॑। वो॒ नमः॑।\\
नमः॒ सेना᳚भ्यो॒ नमः॑। से॒ना॒निभ्य॑श्च॒ नमः॑। वो॒ नमः॑। \\
नमः॑ क्ष॒त्तृभ्यो॒ नमः॑। स॒ङ्ग्र॒ही॒तृभ्य॑श्च॒ नमः॑। वो॒ नमः॑। \\
नम॒स्तक्ष॑भ्यो॒ नमः॑। र॒थ॒का॒रेभ्य॑श्च॒ नमः॑। वो॒ नमः॑। \\
नमः॒ कुला॑लेभ्यो॒ नमः॑। क॒र्मारे᳚भ्यश्च॒ नमः॑। वो॒ नमः॑। \\
नमः॑ पु॒ञ्जिष्टे᳚भ्यो॒ नमः॑। नि॒षा॒देभ्य॑श्च॒ नमः॑। वो॒ नमः॑। \\
नम॑ इषु॒कृद्भ्यो॒ नमः॑। ध॒न्व॒कृद्भ्य॑श्च॒ नमः॑। वो॒ नमः॑।\\
नमो॑ मृग॒युभ्यो॒ नमः॑। श्व॒निभ्य॑श्च॒ नमः॑। वो॒ नमः॑। \\
नमः॒ श्वभ्यो॒ नमः॑। श्वप॑तिभ्यश्च॒ नमः॑। वो॒ नमः॑॥ \\


नमो॑ भ॒वाय॑ च॒ नमः॑। रु॒द्राय॑ च॒ नमः॑। \\
नमः॑ श॒र्वाय॑ च॒ नमः॑। प॒शु॒पत॑ये च॒ नमः॑।\\
नमो॒ नील॑ग्रीवाय च॒ नमः॑। शि॒ति॒कण्ठा॑य च॒ नमः॑। \\
नमः॑ कप॒र्दिने॑ च॒ नमः॑। व्यु॑प्तकेशाय च॒ नमः॑।\\
नमः॑ सहस्रा॒क्षाय॑ च॒ नमः॑। श॒तध॑न्वने च॒ नमः॑। \\
नमो॑ गिरि॒शाय॑ च॒ नमः॑। शि॒पि॒वि॒ष्टाय॑ च॒ नमः॑।\\
नमो॑ मी॒ढुष्ट॑माय च॒ नमः॑। इषु॑मते च॒ नमः॑। \\
नमो᳚ ह्र॒स्वाय॑ च॒ नमः॑। वा॒म॒नाय॑ च॒ नमः॑।\\
नमो॑ बृह॒ते च॒ नमः॑। वर्षी॑यसे च॒ नमः॑। \\
नमो॑ वृ॒द्धाय॑ च॒ नमः॑। सं॒वृध्व॑ने च॒ नमः॑। \\
नमो॒ अग्रि॑याय च॒ नमः॑। प्र॒थ॒माय॑ च॒ नमः॑। \\
नम॑ आ॒शवे॑ च॒ नमः॑। अ॒जि॒राय॑ च॒ नमः॑।\\
नमः॒ शीघ्रि॑याय च॒ नमः॑। शीभ्या॑य च॒ नमः॑। \\
नम॑ ऊ॒र्म्या॑य च॒ नमः॑। अ॒व॒स्व॒न्या॑य च॒ नमः॑। \\
नमः॑ स्त्रोत॒स्या॑य च॒ नमः॑। द्वीप्या॑य च॒ नमः॑॥\\


नमो᳚ ज्ये॒ष्ठाय॑ च॒ नमः॑। क॒नि॒ष्ठाय॑ च॒ नमः॑। \\
नमः॑ पूर्व॒जाय॑ च॒ नमः॑। अ॒प॒र॒जाय॑ च॒ नमः॑। \\
नमो॑ मध्य॒माय॑ च॒ नमः॑। अ॒प॒ग॒ल्भाय॑ च॒ नमः॑। \\
नमो॑ जघ॒न्या॑य च॒ नमः॑। बुध्नि॑याय च॒ नमः॑।\\
नमः॑ सो॒भ्या॑य च॒ नमः॑। प्र॒ति॒स॒र्या॑य च॒ नमः॑। \\
नमो॒ याम्या॑य च॒ नमः॑। क्षेम्या॑य च॒ नमः॑। \\
नम॑ उर्व॒र्या॑य च॒ नमः॑। खल्या॑य च॒ नमः॑। \\
नमः॒ श्लोक्या॑य च॒ नमः॑। अ॒व॒सा॒न्या॑य च॒ नमः॑। \\
नमो॒ वन्या॑य च॒ नमः॑। कक्ष्या॑य च॒ नमः॑। \\
नमः॑ श्र॒वाय॑ च॒ नमः॑। प्र॒ति॒श्र॒वाय॑ च॒ नमः॑। \\
नम॑ आ॒शुषे॑णाय च॒ नमः॑। आ॒शुर॑थाय च॒ नमः॑। \\
नमः॒ शूरा॑य च॒ नमः॑। अ॒व॒भि॒न्द॒ते च॒ नमः॑। \\
नमो॑ व॒र्मिणे॑ च॒ नमः॑। व॒रू॒थिने॑ च॒ नमः॑। \\
नमो॑ बि॒ल्मिने॑ च॒ नमः॑। क॒व॒चिने॑ च॒ नमः॑। \\
नमः॑ श्रु॒ताय॑ च॒ नमः॑। श्रु॒त॒से॒नाय॑ च॒ नमः॑॥ \\


नमो॑ दुन्दु॒भ्या॑य च॒ नमः॑। आ॒ह॒न॒न्या॑य च॒ नमः॑। \\
नमो॑ धृ॒ष्णवे॑ च॒ नमः॑। प्र॒मृ॒शाय॑ च॒ नमः॑।\\
नमो॑ दू॒ताय॑ च॒ नमः॑। प्रहि॑ताय च॒ नमः॑। \\
नमो॑ निष॒ङ्गिणे॑ च॒ नमः॑। इ॒षु॒धि॒मते॑ च॒ नमः॑।\\
नम॑स्ती॒क्ष्णेष॑वे च॒ नमः॑। आ॒यु॒धिने॑ च॒ नमः॑। \\
नमः॑ स्वायु॒धाय॑ च॒ नमः॑। सु॒धन्व॑ने च॒ नमः॑।\\
नमः॒ स्रुत्या॑य च॒ नमः॑। पथ्या॑य च॒ नमः॑। \\
नमः॑ का॒ट्या॑य च॒ नमः॑। नी॒प्या॑य च॒ नमः॑।\\
नमः॒ सूद्या॑य च॒ नमः॑। स॒र॒स्या॑य च॒ नमः॑। \\
नमो॑ ना॒द्याय॑ च॒ नमः॑। वै॒श॒न्ताय॑ च॒ नमः॑।  \\
नमः॒ कूप्या॑य च॒ नमः॑। अ॒व॒ट्या॑य च॒ नमः॑। \\
नमो॒ वर्ष्या॑य च॒ नमः॑। अ॒व॒र्ष्याय॑ च॒ नमः॑। \\
नमो॑ मे॒घ्या॑य च॒ नमः॑। वि॒द्यु॒त्या॑य च॒ नमः॑।\\
नम॑ ई॒ध्रिया॑य च॒ नमः॑। आ॒त॒प्या॑य च॒ नमः॑।\\
नमो॒ वात्या॑य च॒ नमः॑। रेष्मि॑याय च॒ नमः॑। \\
नमो॑ वास्त॒व्या॑य च॒ नमः॑। वा॒स्तु॒पाय॑ च॒ नमः॑॥ \\


नमः॒ सोमा॑य च॒ नमः॑। रु॒द्राय॑ च॒ नमः॑। \\
नम॑स्ता॒म्राय॑ च॒ नमः॑। अ॒रु॒णाय॑ च॒ नमः॑।\\
नमः॑ श॒ङ्गाय॑ च॒ नमः॑। प॒शु॒पत॑ये च॒ नमः॑। \\
नम॑ उ॒ग्राय॑ च॒ नमः॑। भी॒माय॑ च॒ नमः॑। \\
नमो॑ अग्रेव॒धाय॑ च॒ नमः॑। दू॒रे॒व॒धाय॑ च॒ नमः॑।\\
नमो॑ ह॒न्त्रे च॒ नमः॑। हनी॑यसे च॒ नमः॑। \\
नमो॑ वृ॒क्षेभ्यो॒ नमः॑। हरि॑केशेभ्यो॒ नमः॑।\\
नम॑स्ता॒राय॒ नमः॑। नमः॑ श॒म्भवे॑ च॒ नमः॑। \\
म॒यो॒भवे॑ च॒ नमः॑। नमः॑ शङ्क॒राय॑ च॒ नमः॑। \\
म॒य॒स्क॒राय॑ च॒ नमः॑। नमः॑ शि॒वाय॑  च॒ नमः॑। \\
शि॒वत॑राय च॒ नमः॑। नम॒स्तीर्थ्या॑य च॒ नमः॑। \\
कूल्या॑य च॒ नमः॑। नमः॑ पा॒र्या॑य च॒ नमः॑। \\
अ॒वा॒र्या॑य च॒ नमः॑। नमः॑ प्र॒तर॑णाय च॒ नमः॑। \\
उ॒त्तर॑णाय च॒ नमः॑। नम॑ आता॒र्या॑य च॒ नमः॑। \\
आ॒ला॒द्या॑य च॒ नमः॑। नमः॒ शष्प्या॑य च॒ नमः॑। \\
फेन्या॑य च॒ नमः॑। नमः॑ सिक॒त्या॑य च॒ नमः॑। \\
प्र॒वा॒ह्या॑य च॒ नमः॑॥ \\


नम॑ इरि॒ण्या॑य च॒ नमः॑। प्र॒प॒थ्या॑य च॒ नमः॑। \\
नमः॑ किꣳशि॒लाय॑ च॒ नमः॑। क्षय॑णाय च॒ नमः॑। \\
नमः॑ कप॒र्दिने॑ च॒ नमः॑। पु॒ल॒स्तये॑ च॒ नमः॑।\\
नमो॒ गोष्ठ्या॑य च॒ नमः॑। गृह्या॑य च॒ नमः॑। \\
नम॒स्तल्प्या॑य च॒ नमः॑। गेह्या॑य च॒ नमः॑। \\
नमः॑ का॒ट्या॑य च॒ नमः॑। ग॒ह्व॒रे॒ष्ठाय॑ च॒ नमः॑।\\
नमो᳚ ह्रद॒य्या॑य च॒ नमः॑। नि॒वे॒ष्प्या॑य च॒ नमः॑। \\
नमः॑ पाꣳस॒व्या॑य च॒ नमः॑। र॒ज॒स्या॑य च॒ नमः॑।\\
नमः॒ शुष्क्या॑य च॒ नमः॑। ह॒रि॒त्या॑य च॒ नमः॑। \\
नमो॒ लोप्या॑य च॒ नमः॑। उ॒ल॒प्या॑य च॒ नमः॑। \\
नम॑ ऊ॒र्व्या॑य च॒ नमः॑। सू॒र्म्या॑य च॒ नमः॑। \\
नमः॑ प॒र्ण्या॑य च॒ नमः॑। प॒र्ण॒श॒द्या॑य च॒ नमः॑। \\
नमो॑ऽपगु॒रमा॑णाय च॒ नमः॑। अ॒भि॒घ्न॒ते च॒ नमः॑। \\
नम॑ आख्खिद॒ते च॒ नमः॑। प्र॒ख्खि॒द॒ते च॒ नमः॑। नमो॑ वो॒ नमः॑। \\
कि॒रि॒केभ्यो॒ नमः॑। दे॒वाना॒ꣳ॒ हृद॑येभ्यो॒ नमः॑। \\
नमो॑ विक्षीण॒केभ्यो॒ नमः॑। नमो॑ विचिन्व॒त्केभ्यो॒ नमः॑। \\
नम॑ आनिर्‌ह॒तेभ्यो॒ नमः॑। नम॑ आमीव॒त्केभ्यो॒ नमः॑। \\

\centerline{॥इति श्री श्रीरुद्रनाम त्रिशती सम्पूर्णा॥}
